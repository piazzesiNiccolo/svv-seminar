\documentclass{beamer}
\usepackage{babel}
\usepackage{hyperref}
\usepackage[utf8]{inputenc}
\usepackage[T1]{fontenc}
\usepackage{tikz}
\usepackage{graphicx}
\usepackage{xcolor}
\usepackage{amsmath}
\usetikzlibrary{
    arrows,
    positioning,
    shapes,
    quotes
}
\newcommand{\nodes}[1]{%
    \foreach \num [count=\n starting from 0] in {#1}{
      \node[minimum size=3mm, draw, circle,fill=black!10] (n\n) at (\n,0) {\textbf{\num}};
    }
}
\AtBeginSection[ ]
{
\begin{frame}{Outline}
    \tableofcontents[currentsection, hideallsubsections]
\end{frame}
}
\title{Separation Logic}
\author{Niccolò Piazzesi}
\institute[UniPi]{
    Università degli studi di Pisa \\
    Anno Accademico 2021-22
}
\begin{document}
    \begin{frame}
        \maketitle
    \end{frame}
    \begin{frame}{Outline}
        \tableofcontents[hideallsubsections]
    \end{frame}
    \section{Introduction}
    \begin{frame}
            \frametitle{Brief recap: reasoning about code}
            
            \begin{itemize}
                \item Program semantics described by logical conditions satisfied by language constructs
                \item Classical model, first put forward by Robert W. Floyd and Tony Hoare
            \end{itemize}
            
        
    \end{frame}
    \begin{frame}
        \frametitle{Floyd-Hoare Logic in 1 slide}
        \begin{center}
            \huge
            \{P\}S\{Q\}
        \end{center}
        \begin{itemize}
            \item P : pre-conditions
            \item S : statement
            \item Q : post conditions
        \end{itemize}
        \bigskip
        
        Partial correctness: \textbf{If the inital state fullfils pre-conditions and the statement terminates}, the final state satisfies the post conditions.
        \medskip

        Total correctness: \textbf{If the initial state fullfils the pre-conditions} then the statement terminates and the final state satisfies the post-conditions.
    \end{frame}

    \begin{frame}
        \frametitle{Limitations}
        
        Does not work for non terminating programs 
        \pause    
        \bigskip
        
        Becomes complex with modular constructs such as objects and unconditional jumps
        \pause
        \bigskip

        \textbf{Global view of state becomes a burden when introducing pointers( think of pointer aliasing..)}

        
    \end{frame}
    \begin{frame}
        \frametitle{Motivating example}
    \end{frame}
    \section{Theoretical Foundations}
    \subsection{The model and assertion language}
    \subsection{SL proofs}
    \subsection{Examples}
    \section{Reasoning with separation logic}
    \subsection{Abstract Interpretation: Biabduction and symbolic heap  analysis}
    \subsection{Model Checking for Symbolic-Heap Separation Logic }
    \section{Tools}
    \subsection{Smallfoot}
    \subsection{Infer}
    \subsection{SLAyer}
    \begin{frame}[allowframebreaks]
     
        \nocite{*}
        \frametitle{References}
        \bibliographystyle{unsrt}
        \bibliography{bib.bib}
    \end{frame}
\end{document}
