\documentclass{beamer}
\usepackage{babel}
\usepackage{hyperref}
\usepackage[utf8]{inputenc}
\usepackage[T1]{fontenc}
\usepackage{tikz}
\usepackage{graphicx}
\usepackage{xcolor}
\usepackage{amsmath}
\usetikzlibrary{
    arrows,
    positioning,
    shapes,
    quotes
}
\newcommand{\nodes}[1]{%
    \foreach \num [count=\n starting from 0] in {#1}{
      \node[minimum size=3mm, draw, circle,fill=black!10] (n\n) at (\n,0) {\textbf{\num}};
    }
}
\AtBeginSection[ ]
{
\begin{frame}{Outline}
    \tableofcontents[currentsection, hideallsubsections]
\end{frame}
}
\title{Separation Logic}
\author{Niccolò Piazzesi}
\institute[UniPi]{
    Università degli studi di Pisa \\
    Anno Accademico 2021-22
}
\begin{document}
    \begin{frame}
        \maketitle
    \end{frame}
    \begin{frame}{Outline}
        \tableofcontents[hideallsubsections]
    \end{frame}
    \section{Introduction}

    \begin{frame}
        ciao \cite{o2001local}
    \end{frame}
    \section{Theoretical Foundations}
    \subsection{The model and assertion language}
    \subsubsection{Theorem proving}
    \subsubsection{Examples}
    \section{Reasoning with separation logic}
    \subsection{Abstract Intepreation: Biabduction and shape analysis}
    \subsection{Model checking}
    \section{Tools}
    \subsection{Smallfoot}
    \subsection{Infer}
    \subsection{SLAyer}
    \begin{frame}[allowframebreaks]
     
        \frametitle{References}
        \bibliographystyle{unsrt}
        \bibliography{bib.bib}
    \end{frame}
\end{document}
